\documentclass[a4paper,12pt]{article}

% Import the deliverable package from common directory
\usepackage{../common/deliverable}

% Tell LaTeX where to find graphics files
\graphicspath{{../common/logos/}{./figures/}{../}}

\usepackage{xspace}
\usepackage{lipsum}

% Set the deliverable number (without the D prefix, it's added automatically)
\setdeliverableNumber{4.4}

% Begin document
\begin{document}

% Create the title page with the title as argument
\maketitlepage{Table of Exploitable Results}

\newpage

% Main Table using the new environment and command
\begin{deliverableTable}
    \tableEntry{Deliverable title}{Table of Exploitable Results}
    \tableEntry{Deliverable number}{D4.4}
    \tableEntry{Deliverable version}{3.0}
    \tableEntry{Date of delivery}{31 March 2025}
    \tableEntry{Actual date of delivery}{28 March 2025}
    \tableEntry{Nature of deliverable}{Report}
    \tableEntry{Dissemination level}{Public}
    \tableEntry{Work Package}{WP4}
    \tableEntry{Partner responsible}{RUB}
\end{deliverableTable}

% Abstract and Keywords Section
\begin{deliverableTable}
    \tableEntry{Abstract}{D4.4 summarizes the exploitable results over the entire project duration and the respective managing partners within the consortium of the dealii-X project.}
    \tableEntry{Keywords}{Exploitable results; }
\end{deliverableTable}

\newpage

\begin{documentControl}
    \addVersion{1.0}{13.03.2025}{I. Prusak, R. Schussnig}{Initial draft}
    \addVersion{2.0}{24.03.2025}{I. Prusak, R. Schussnig}{Information gathered from all partners}
    % \addVersion{0.3}{[Date]}{[Author name]}{[Description of changes]}
    \addVersion{3.0}{26.03.2025}{I. Prusak, R. Schussnig}{Final version}
\end{documentControl}

\subsection*{{Approval Details}}
Approved by: Martin Kronbichler \\
Approval Date: 26.03.2025

\subsection*{{Distribution List}}
\begin{itemize}
    \item [] - Project Coordinators (PCs)
    \item [] - Work Package Leaders (WPLs)
    \item [] - Steering Committee (SC)
    \item [] - European Commission (EC)
\end{itemize}

\vspace*{2cm}

\disclaimer

\newpage

\tableofcontents % Automatically generated and hyperlinked Table of Contents

\newpage

\section{{Introduction}} \label{sec:introduction}

The CoE dealii-X project, titled ``Exascale Framework for Digital Twins of the Human Body'', seeks to transform computational methods in medical science by developing a high-performance exascale computing framework. Using the deal.II library, it aims to create highly detailed digital twins of human organs, enabling real-time simulations of complex biological processes.

The project follows a separation-of-concerns approach, with different components managed by specialized teams to meet the needs of various domain experts. Key efforts include optimizing deal.II for exascale performance by improving core computational algorithms, enhancing node-level efficiency, and addressing challenges like scalable linear algebra routines and efficient mesh handling. Additionally, it focuses on co-designing for emerging HPC architectures, ensuring energy efficiency and maximizing hardware capabilities.

Beyond technical advancements, community engagement, training, and knowledge dissemination play a crucial role, fostering collaboration within the HPC ecosystem. The project is organized into five work packages (WPs) covering method and algorithmic development, modeling, computation, validation, and dissemination.

The consortium assembled for this project brings together a diverse and complementary range of expertise, facilities, and experiences necessary to achieve the ambitious objectives of the project. Each participating institution contributes unique strengths that collectively form a robust and interdisciplinary team capable of tackling the complex challenges of exascale computing and digital twin development. The related key exploitable results from WPs 1 to 3 are summarized in Tables~\ref{tab:exploitable_results_WP1}, \ref{tab:exploitable_results_WP2}, ~\ref{tab:exploitable_results_WP3} and~\ref{tab:exploitable_results_WP4}.

\newpage

\section{{Table of Exploitable Results}}~\label{sec:table_exploitable_results}

WP1: \textit{Exascale Building Blocks and Support Tools} focuses on upgrading the deal.II library for exascale computing. Key tasks include developing advanced GPU algorithms, optimizing performance at both node and full-machine levels, integrating with exascale libraries and tools, and refining coupling algorithms for complex multiphysics simulations. The related exploitable results are given in Tab.~\ref{tab:exploitable_results_WP1}.

\begin{center}
    \small
    \renewcommand{\arraystretch}{1.25}
    \begin{longtable}{|l|p{2.5cm}|p{12cm}|}
    \caption{Exploitable results related to the dealii-X project and respective managing partners within WP1.}
    \label{tab:exploitable_results_WP1}
    \\
    \hline
    \textbf{Nr.} & \textbf{Manager} & \textbf{Description} \\
    \hline
    1 & RUB &
    Run multigrid solver with deal.II library on full JUPITER system,
    quantify its runtime performance in terms of achieved FLOP/s rate, 
    achieved memory bandwidth, and strong/weak scaling behavior, and
    by comparing the performance with various PSCToolkit und MUMPS
    solver components on representative geometry of the dealii-X applications.
    \\
    \hline
    2 & RUB &
    Run implicit solver for coupled flow-transport problem in one of the application areas (lung, heart, brain) on supercomputer scale, comparing against today's implicit/explicit schemes, including scalability and floating point and memory access efficiency 
    \\
    \hline
    3 & UNIPI &
    Extend the Gmsh API in deal.II to enable the generation of highly complex meshes optimized for exascale simulations. Leverage the new generalized coupling operator interface to streamline multiphysics and multiscale problem setups. Benchmark the framework’s efficiency and scalability on supercomputer-scale applications, demonstrating its impact on high-fidelity simulations in biomedical and engineering domains.
    \\
    \hline
    4 & UNIPI &
    Extend deal.II’s reduced-order modeling (ROM) and low-rank approximation methods to enable highly efficient, exascale-compatible simulations of patient-specific organ behavior. Validate the approach on biomedical applications such as brain, lung, and cardiovascular simulations, demonstrating improved computational performance and potential for real-time clinical insights. 
    \\
    \hline
    5 & SISSA &
    Extend deal.II library to support polygonal and polyhedral meshes aiming at complexity reduction in complex multiscle problems. Develop proof-of-concept polygonal discretisation examples in fluid mechanics. Exploit polygonal discretization methods as efficient coarse grid solvers.
    \\
    \hline
    6 & SISSA &
    Develop a multigrid solver for electrophysiology problems within deal.II, leveraging the polytopal agglomeration strategies to dynamically generate efficient polygonal grid hierarchies. This approach accelerates the solution of complex electrophysiological models on general meshes, enhancing computational performance in large-scale heart modeling and other biomedical applications.  
    \\
    \hline
    7 & UNITOV, CNR &
    Develop an integrated interface between deal.II and PSCToolkit to simplify access to advanced linear solvers and preconditioners. Optimize the framework for GPU architectures to accelerate large-scale simulations, improving computational efficiency in applications such as human organ modeling and other complex multiphysics problems. Benchmark performance against existing CPU-based approaches to demonstrate scalability and robustness.
    \\
    \hline
    8 & INPT, CNRS &
    Integrate the MUMPS direct solver within deal.II to provide a robust and efficient solution for large-scale sparse linear systems. Enable its use in multigrid and domain decomposition methods as a coarse-grid solver to improve convergence in complex coupled problems. Optimize MUMPS for GPU acceleration and implement low-rank approximations (BLR) and mixed-precision strategies to enhance computational efficiency and reduce memory requirements in large-scale simulations.
    \\
    \hline
    \end{longtable}
%    \end{table}
\end{center}

\newpage

WP2: \textit{Lighthouse Applications for Digital Twins} focuses on scaling advanced organ simulation models—such as those for the lungs, heart, brain, and liver—to exascale computing. The aim is to achieve unprecedented detail in simulations, enhancing medical research and personalized medicine. These models will serve as prototypes for developing simulation methods for additional organs.

\begin{center}
    \small
    \renewcommand{\arraystretch}{1.25}
    \begin{longtable}{|l|p{2.5cm}|p{12cm}|}
    \caption{Exploitable results related to the dealii-X project and respective managing partners within WP2.}
    \label{tab:exploitable_results_WP2}
    \\
    \hline
    \textbf{Nr.} & \textbf{Manager} & \textbf{Description} \\
    \hline
    9 & TUM &
    Run simulations on realistic reolved lung parenchyma geometries with matrix-free geometric multigrid solvers. Enable new insights into effect of local surfactant deficiencies on lung mechanics and on that basis derive reduced-order formulation for bedside application.
    \\
    \hline
    10 & POLIMI &
    Perform biophysically detailed heart simulations in lifex with matrix-free-based solver. Quantify their performance in terms of computational time to solution and parallel scalability, and compare these results against the current (matrix-based) implementation.
    \\
    \hline
    11 & FAU &
    Prepare poro-viscoelastic simulation solver to predict the response of brain tissue for high-performance computing environments. Inversely identify poro-viscoelastic parameters for different regions of the human brain based on large-strain mechanical measurement data.
    \\
    \hline
    12 & FVB-WIAS &
    Simulation of multiscale tissue coupled with complex one-dimensional vascular network models.
    \\
    \hline
    13 & FVB-WIAS &
    Utilization of the multiscale model for the inverse estimation of effective mechanical parameters in liver tissue. 
    \\
    \hline
    14 & UNIBS &
    Developed code for mixture theory coupled with mechanics, to model and simulate blood clotting and cellular motility. Manufactured solution tests run.
    \\
    \hline
    15 & EXACT-LAB, Dualistic &
    Low-code platform for multi-tiered definition/interaction with advanced simulation of human organs
    \\
    \hline
    16 & EXACT-LAB, Dualistic &
    Backend and Meta-scheduler for transparent job submission of human organ simulations on large HPC cluster 
    \\
    \hline
    \end{longtable}
%    \end{table}
\end{center}

\newpage

WP3: \textit{Co-Design, Technology Exploitation \& Energy Efficiency} focuses on optimizing hardware and software for maximum efficiency and performance in exascale systems. It emphasizes energy efficiency through collaboration with European institutions and industry partners. The WP also includes benchmarking activities to assess and refine algorithm and application performance under real-world conditions.

\begin{center}
    \small
    \renewcommand{\arraystretch}{1.25}
    \begin{longtable}{|l|p{2.5cm}|p{12cm}|}
    \caption{Exploitable results related to the dealii-X project and respective managing partners within WP3.}
    \label{tab:exploitable_results_WP3}
    \\
    \hline
    \textbf{Nr.} & \textbf{Manager} & \textbf{Description} \\
    \hline
    17 & BADW-LRZ &
    Optimized Computational Kernels: Improve deal.II performance across diverse architectures through parallel programming models, compiler optimizations, hardware-specific software tuning, and use of vendor-specific performance libraries.
    \\
    \hline
    18 & BADW-LRZ &
    Scalable Benchmark Suite: Provide cluster- and node-level benchmarks for both finite-element methods (including hardware roofline analysis, energy measurements, and frequency scaling) and coupled algorithms to optimize scalability, energy efficiency, and computational performance.
    \\
    \hline
    19 & BADW-LRZ &
    Training Opportunities and Knowledge Transfer: LRZ provides training courses designed to enhance market readiness and workforce skills in programming languages and HPC-related parallel programming models. Recent additions include deep learning and AI topics, reflecting the growing commercial demand for expertise in these fields. Wherever feasible, course materials are openly licensed to promote widespread adoption, while tailored offerings can address the needs of dealii-X project partners and stakeholders.
    \\
    \hline
    \end{longtable}
%    \end{table}
\end{center}

\newpage

WP4: \textit{Dissemination, Communication, and Exploitation} focuses on training, community engagement, and knowledge sharing within the HPC ecosystem. It aims to spread best practices, communicate project findings, and promote exascale technologies. Additionally, this WP ensures that the developed innovations are effectively utilized and reach key stakeholders, maximizing the project's impact.

\begin{center}
    \small
    \renewcommand{\arraystretch}{1.25}
    \begin{longtable}{|l|p{2.5cm}|p{12cm}|}
    \caption{Exploitable results related to the dealii-X project and respective managing partners within WP4.}
    \label{tab:exploitable_results_WP4}
    \\
    \hline
    \textbf{Nr.} & \textbf{Manager} & \textbf{Description} \\
    \hline
    20 & VPHI &
    Communicate the project's goals, progress and impact to the broader public for different target audiences including website, printed media, content on social-media platforms, and a publication repository which will be updated regularly with project news and achievements, thus contributing to the growth of dedicated communities of stakeholders.
    \\
    \hline
    \end{longtable}
%    \end{table}
\end{center}

The remaining WP5 \textit{Management} focuses on data management, coordination, and administration across the entire project. It ensures that a comprehensive data management plan is implemented and maintained throughout, while also overseeing the technical and financial management of all work packages to ensure the project progresses smoothly and successfully. Even though the successful execution of this WP is vital for the overall project, no exploitable results are tied to it.

\section{{Conclusion}} \label{sec:conclusion}

The dealii-X project contributes to advancing exascale computing for digital twins of the human body. The exploitable results summarized within this document hence comprise several aspects of i) enhancing the deal.II library for exascale performance, ii) scaling organ simulations, such as those for the lungs, heart, and brain, to exascale, iii) addressing hardware-software co-design and energy efficiency optimization, iv) dissemination, communication, and community engagement, ensuring knowledge transfer and the utilization of developed technologies. These efforts collectively aim to maximize exploitable results, ensuring the project’s impact in the field of high-performance computing.

\label{MyLastPage}


\end{document}
