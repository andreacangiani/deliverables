\documentclass[a4paper,12pt]{article}

% Import the deliverable package from common directory
\usepackage{../common/deliverable}

% Tell LaTeX where to find graphics files
\graphicspath{{../common/logos/}{./figures/}{../}}

\usepackage{xspace}
\usepackage{lipsum}

% Set the deliverable number (without the D prefix, it's added automatically)
\setdeliverableNumber{1.1}

% Begin document
\begin{document}

% Create the title page with the title as argument
\maketitlepage{External libraries integration}

\newpage

% Main Table using the new environment and command
\begin{deliverableTable}
    \tableEntry{Deliverable title}{External libraries integration}
    \tableEntry{Deliverable number}{D1.1}
    \tableEntry{Deliverable version}{0.1}
    \tableEntry{Date of delivery}{31/07/2025}
    \tableEntry{Actual date of delivery}{\today}
    \tableEntry{Nature of deliverable}{Other}
    \tableEntry{Dissemination level}{Public}
    \tableEntry{Work Package}{WP1}
    \tableEntry{Partner responsible}{INPT}
\end{deliverableTable}

% Abstract and Keywords Section
\begin{deliverableTable}
    \tableEntry{Abstract}{This document describes the integration of
        external libraries in the deal.II finite element library, namely
        the MUMPS sparse direct solver, the PSCToolkit iterative solver
        and the GMSH three-dimensional finite element mesh generator.}
    \tableEntry{Keywords}{API; solvers; mesh; parallelism; interfaces}
\end{deliverableTable}

\newpage

\begin{documentControl}
    \addVersion{0.1}{15/07/2025}{Alfredo Buttari}{Initial draft}
    \addVersion{0.2}{22/07/2025}{Marco Feder}{Preliminary experimental results}
    \addVersion{0.3}{[Date]}{[Author name]}{[Description of changes]}
    \addVersion{1.0}{[Date]}{[Author name]}{Final version}
\end{documentControl}

\subsection*{{Approval Details}}
Approved by: [Name] \\
Approval Date: [Date]

\subsection*{{Distribution List}}
\begin{itemize}
    \item [] - Project Coordinators (PCs)
    \item [] - Work Package Leaders (WPLs)
    \item [] - Steering Committee (SC)
    \item [] - European Commission (EC)
\end{itemize}

\vspace*{2cm}

\disclaimer

\newpage

\tableofcontents % Automatically generated and hyperlinked Table of Contents

\newpage

\section{{Introduction}}

The objective of the dealii-X Work Package 1 is to enhance the
capabilities of the deal.ii finite element library through the use of
some external libraries, namely, the MUMPS and PSCToolkit solvers and
the GMSH three-dimensional finite element mesh generator. MUMPS is a
parallel direct solver for sparse linear systems, i.e., it achieves
the solution by computing the factorization (LU, $LDL^T$ or Cholesky,
depending of the problem type) of the system matrix and using the
resulting factors in forward/backward substitution operations. One of
the objectives of WP1 is to assess the effectiveness of some of the
recent advanced features of MUMPS such as the block low-rank
approximations or the mixed-precision computations on problems issued
from the simulation of human body organs. PSCToolkit implements
multiple parallel iterative solvers of the Krylov family for the
solution of large scale sparse linear systems. These methods are, by
nature, scalable; furthermore, PSCToolkit provides efficient algebraic
multigrid preconditioners to improve the convergence. Finally, WP1
deals with the integration of the GMSH three-dimensional finite
element meshing package in order to optimize the mesh generation at
large scale.

\subsection{{Purpose of the Document}}
This document describes the integration of the above-mentioned
external packages, the corresponding API within the deal.ii framework
and the features that are exposed to enable experimenting with
advanced features in the context of the dealii-X project.

% \subsection{Objectives of Work Package 1 (WP1)}

% The main objective of Work Package 1 (WP1) is to serve as the foundation
% for the dealii-X Centre of Excellence by enhancing and expanding the
% capabilities of the deal.II library to address the challenges of exascale
% computing and facilitate the creation of advanced digital twins of human
% organs.

% The key steps of WP1 include:
% \begin{itemize}
% \item Extending and improving the exascale capabilities of deal.II;
% \item Improving pre-exascale modules of the deal.II library;
% \item Developing an experimental polygonal discretization module for
%   deal.II;
% \item Integrating PSCToolkit within deal.II;
% \item Integrating MUMPS within deal.II.
% \end{itemize}

% Specifically, the sub-work packages aim to:
% \begin{itemize}
% \item \textbf{WP1.1 (Lead RUB)}: Develop matrix-free computational
%   methods optimized for GPU architectures and enhance the scalability
%   of solvers;
%     \item \textbf{WP1.2 (Lead UNIPI)}: Improve the gmsh API, develop a
%       generalized interface for coupling operators, enhance reduced
%       order modelling capabilities, integrate low-rank approximation
%       methods, and develop block preconditioners;
%     \item \textbf{WP1.3 (Lead SISSA)}: Introduce and parallelize
%       polygonal discretization methods within deal.II and develop
%       related multigrid techniques;
%     \item \textbf{WP1.4 (Lead UNITOV)}: integrate PSCToolkit into
%       deal.II, leveraging GPU computing and developing efficient
%       preconditioners for multiphysics problems;
%     \item \textbf{WP1.5 (Lead INPT)}: Integrate the MUMPS solver
%       directly into deal.II for use in multigrid methods and explore
%       low-rank and mixed-precision techniques;
% \end{itemize}

% In summary, WP1 is dedicated to developing and integrating fundamental
% software components within the deal.II library and external libraries,
% with a strong emphasis on enabling exascale computation for the
% digital twin applications in WP2.

% \subsection{Purpose and Scope of this Report (Deliverable D1.1)}

% The purpose of this report is to make a comprehensive analysis of the
% development that are necessary to integrate external libraries,
% namely, MUMPS, PSCToolkit and GMSH, within the deal.II package. 


% \subsection{{Structure of the Document}}
% \begin{itemize}
% \item Section \ref{sec:section2}: [Section Title]
% \item Section \ref{sec:section3}: [Section Title]
% \item Section \ref{sec:section4}: [Section Title]
% \item Section \ref{sec:section5}: [Section Title]
% \item Section \ref{sec:section6}: [Section Title]
% \end{itemize}

\newpage

\section{MUMPS}
\label{sec:section2}

\subsection{Existing interface}

The deal.II package used to support a direct MUMPS integration in the
past. This support was subsequently remove in favor of an indirect use
of MUMPS through the PETSc library. As a first step towards the
objectives of WP1 of the dealii-X project, this support was reverted,
as documented in Pull Request \#18255
(\url{https://github.com/dealii/dealii/pull/18255}) which has been
subsequently merged in the main branch. This minimalistic interface
includes a number of basic tests.


\subsection{Implemented improvements in the MUMPS support}

The basic deal.II MUMPS interface did not include support for
distributed-memory parallelism, which means that the system matrix and
right-hand side(s) were entirely assembled on the master process where
the subsequent phases (symbolic analysis, factorization and
backward/forward substitution) take place; not only this might be
infeasible due to memory limitations but it severely limits the
performance of the sparse direct solver. The MUMPS integration in
deal.ii was extended to support distributed memory parallelism through
the use of distributed matrix and vector datatypes as defined in the
deal.ii PETSc and Trilinos wrappers. This allows for a completely
parallel initialization of the MUMPS data structure (that includes the
system matrix and the right-hand sides) prior to the symbolic
analysis, numerical factorization and forward/backward substitution.

Setting-up MUMPS internal configuration parameters was not possible in
original deal.II MUMPS interface which, therefore, had to be
extended to enable the use of the more advanced features of the MUMPS
solver; these include the Block Low-Rank approximations, the GPU
support and numerous other parameters that allow for fine-tuning both
the performance and the numerical robustness of the solver. This
extension was achieved by adding a \texttt{AdditionalData} object
to the constructor of the MUMPS solver class that takes the desired
values for selected solver parameters.

The work related to above-mentioned extension is documented in pull
request \#18497 (\url{https://github.com/dealii/dealii/pull/18497})
which was eventually merged in the master deal.ii branch.

Furthermore, the deal.ii documentation was updated with a detailed description
of the MUMPS API within deal.ii:

\url{https://dealii.org/developer/doxygen/deal.II/classSparseDirectMUMPS.html}

\subsection{Preliminary experimental results}
The new MUMPS interface has been validated through a series of tests
to verify its correctness, using both serial and distributed sparse matrices. Such tests
have been added to the deal.ii test-suite and are run automatically at
each commit. In particular, the following additional steps have
been carried out to validate the robustness of the current interface:
\begin{itemize}
    \item Usage of Block-Low Rank approximations as a preconditioner
          for a diffusion problem in 2D and 3D, on a sequence of parallel and
          adaptively refined meshes, both for PETSc and Trilinos distributed matrices, thereby
          confirming the correctness of the distributed memory parallelism
          support and the capabilities of BLR as a preconditioner.
    \item Integration into the \emph{Brain application}, as added
          in pull request \url{https://github.com/BRAINIACS-Group/ExaBrain/pull/1}, serving as a
          drop-in replacement for the previous solver based on \texttt{Amesos\_Superludist}
          from Trilinos. Preliminary results across several
          parameter configurations show that the new MUMPS solver
          reduces the total solver time by a factor
          ranging from 2 up to approximately 3.

\end{itemize}
\newpage

\section{PSCToolkit}
\label{sec:section3}

\lipsum[8-9]

\subsection{{[Subsection Title]}}
\begin{itemize}[left=1em, itemsep=0pt, topsep=0pt]
    \item \lipsum[10][1-2]
    \item \lipsum[10][3-4]
    \item \lipsum[10][5-6]
\end{itemize}

\subsection{{[Subsection Title]}}
\lipsum[11]

\newpage

\section{GMSH}
\label{sec:section4}

\lipsum[12-13]

\newpage

\section{{Conclusion}} \label{sec:conclusion}

\lipsum[20]

\label{MyLastPage}

\end{document}

%%% Local Variables:
%%% mode: LaTeX
%%% TeX-master: t
%%% End:
