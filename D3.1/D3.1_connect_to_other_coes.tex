
\documentclass[a4paper,12pt]{article}

% Import the deliverable package from common directory
\usepackage{../common/deliverable}

% Tell LaTeX where to find graphics files
\graphicspath{{../common/logos/}{./figures/}{../}}

\usepackage{xspace}
\usepackage{lipsum}

% Set the deliverable number (without the D prefix, it's added automatically)
\setdeliverableNumber{3.1}

% Begin document
\begin{document}

% Create the title page with the title as argument
\maketitlepage{Connection to other CoEs}

\newpage

% Main Table using the new environment and command
\begin{deliverableTable}
    \tableEntry{Deliverable title}{Connection to other CoEs}
    \tableEntry{Deliverable number}{D3.1}
    \tableEntry{Deliverable version}{4.0}
    \tableEntry{Date of delivery}{31 March 2025}
    \tableEntry{Actual date of delivery}{31 March 2025}
    \tableEntry{Nature of deliverable}{Report}
    \tableEntry{Dissemination level}{Public}
    \tableEntry{Work Package}{WP3}
    \tableEntry{Partner responsible}{BADW-LRZ}
  \end{deliverableTable}



% Abstract and Keywords Table
\begin{deliverableTable}
  \tableEntry{Abstract}{D3.1 presents the integration of deal.II into EuroHPC's overarching infrastructure CASTIEL 2}
  \tableEntry{Keywords}{supercomputer access in EU; continuous integration; continuous deployment}
\end{deliverableTable}


\begin{documentControl}
    \addVersion{1.0}{20.03.2025}{M. Kronbichler}{Initial Draft}
    \addVersion{2.0}{27.03.2025}{I. Pribec, G. Mathias, M. Allalen}{Augment Document}
    \addVersion{3.0}{27.03.2025}{M. Kronbichler}{Minor Additions}
    \addVersion{4.0}{27.03.2025}{I. Pribec}{Final Version}
\end{documentControl}

\subsection*{{Approval Details}}
Approved by: Martin Kronbichler \\
Approval Date: 31.03.2025

\subsection*{{Distribution List}}
\begin{itemize}
    \item [] - Project Coordinators (PCs)
    \item [] - Work Package Leaders (WPLs)
    \item [] - Steering Committee (SC)
    \item [] - European Commission (EC)
\end{itemize}

\vspace*{2cm}

\disclaimer

\newpage


\newpage
\section{dealii-X within CASTIEL 2}

The dealii-X project is a Centre of Excellence (CoE) funded in as a project in
the European High-Performance Computing Joint Undertaking under the powers
delegated by the European Commission.

The use of supercomputers is at the heart of the dealii-X project. In the
European Union, 10 CoEs were started in 2023, and 4 additional CoEs started in
the two subsequent years. These projects share the goal to create codes that
efficiently utilize modern high-end supercomputers up to emerging exascale
systems (including EuroHPC systems) and to make these codes available to the
relevant user groups. The dealii-X project share these goals, and shares the
challenges of many other projects in terms of porting effort for our codes to
ensure scalability on each target system. Given the similar tasks faced by
several projects, the EuroHPC centres of excellence are coordinated through
the project CASTIEL 2, the Coordination \& Support for National Competence
Centres on a European Level Phase 2, project number 101102047, who state
their mission
as\footnote{\url{https://eurohpc-ju.europa.eu/research-innovation/our-projects/castiel-2_en}}\footnote{\url{https://www.eurocc-access.eu}}
\begin{center}
  \begin{minipage}{0.8\textwidth}
    The project encourages collaboration, exchange of knowledge between HPC
    practitioners, expertise and best practices between Participating States
    and the mapping of skills and competences across Europe to support
    cohesion and a more consistent level of expertise across Europe. In this
    next phase of the project, the CASTIEL 2 project also takes on an
    additional role of providing similar coordination support to the EuroHPC
    Centres of Excellence (CoEs).

    The CoEs gather HPC expertise in the development of HPC applications by
    scientific domain at a European level, CASTIEL 2 will build a stronger HPC
    community which will foster strategic collaboration in HPC research and
    deployment of skills and in expertise in HPC technologies and applications
    between CoEs and NCCs.
  \end{minipage}

  \begin{minipage}{0.8\textwidth}
    The objective of CASTIEL 2 and EuroCC 2 is to strengthen the European
    Union (EU)’s technological autonomy and competitiveness by ensuring a
    coordinated and consistent high level of knowledge sharing across Europe
    in HPC and related disciplines such as high-performance data analytics
    (HPDA) and HPC-based artificial intelligence.
  \end{minipage}
\end{center}

CASTIEL 2 has the role of coordinating the work of the various CoEs. Among the
CoEs, around 50 codes for scientific computations are developed and
maintained, which differ vastly in size, maturity, scalability and age. As a
CoE, dealii-X has picked up on various levels of integration with the CASTIEL
2 project as a coordinating agency, as well as with other CoEs in terms of
similar scientific strategies. The actions of these steps are described on the
following pages.

\section{Online presentation of CASTIEL 2}

In exchange with the partners at CASTIEL 2, the dealii-X project has delivered
their input to various web pages. On \url{https://www.hpccoe.eu/}, we are now
listed as a
project\footnote{\url{https://www.hpccoe.eu/eu-hpc-centres-of-excellence2/dealii-x/}}
with the main project description

\begin{center}
  \begin{minipage}{0.8\textwidth}
    \textcolor{EUblue}{dealii-X} aims at developing a scalable,
    high-performance computational platform to create accurate digital twins
    of human organs using the deal.II finite element library. The framework
    will combine sophisticated numerical methods with exascale computing
    capabilities to create cutting-edge multiphysics and multidisciplinary
    simulation models. Lighthouse applications representing crucial processes
    in the human brain, the cardiovascular and respiratory systems as well as
    the liver will be tackled to gain new insights into biological processes
    of the human body and aiding in personalized medicine. A primary focus is
    on a new generation of solvers, including novel methods for large linear,
    nonlinear and coupled systems, matrix-free finite element algorithms for
    GPUs and discretization schemes.
  \end{minipage}
\end{center}

We have integrated the dealii-X web page at \url{https://www.dealii-X.eu},
where we report on the recent developments of the dealii-X project in terms of
codes, scientific achievements, and prizes, to just name a few. We have
clarified this involvement with CASTIEL 2 and are committed to provide future
updates as requested by our partners. One point of contact is the interaction
of dealii-X with partners from Ginkgo, who are active in the MICROCARD-2
project.

As part of these effort, we have contributed to Work Package 2 of CASTIEL 2
concerning the summary of the consortium, objectives, area of activity, codes
and use cases.

\section{Integration of dealii-X codes into WP2 of CASTIEL~2}

In December 2024 to February 2025, we have intensively collaborated with Work
Package 2 of CASTIEL 2 concerning the codes and use cases. We have presented
our three main mathematical codes, \href{https://github.com/dealii/dealii}{\texttt{deal.II}}, \href{https://psctoolkit.github.io/libraries/}{\texttt{PSCToolkit}} and
\href{https://mumps-solver.org/index.php}{\texttt{MUMPS}} for the CoE codes whitebook version 1.3, using the following
descriptions:


{\small
  \noindent
  \renewcommand{\arraystretch}{2.5}
\begin{tabular}{llll}
  \hline
  \textbf{Code} & \textbf{Category} & \textbf{Description} & \textbf{Main uses} \\
  \hline
  \href{https://github.com/dealii/dealii}{deal.II} & \begin{minipage}{0.1\textwidth}Multi-physics\end{minipage} & \begin{minipage}{0.4\textwidth}
    Wide range of finite element algorithms, multigrid methods, matrix-free and matrix-based algorithms
  \end{minipage}
                                & \begin{minipage}{0.26\textwidth}Discretization of partial differential equations\end{minipage}
  \\
  \href{https://psctoolkit.github.io/libraries/}{PSCToolkit} & \begin{minipage}{0.1\textwidth}Linear solver\end{minipage} &\begin{minipage}{0.4\textwidth}
  Framework for solving large and sparse linear systems
  \end{minipage}
                                & \begin{minipage}{0.26\textwidth}Linear algebra applications\end{minipage}
  \\
  \href{https://mumps-solver.org/index.php}{MUMPS} & \begin{minipage}{0.1\textwidth}Linear solver\end{minipage} &\begin{minipage}{0.4\textwidth}
  A numerical software package for solving sparse systems of linear equations
  \end{minipage}
                                                           & \begin{minipage}{0.26\textwidth}Linear algebra applications\end{minipage}
  \\
  \hline
\end{tabular}
}

The provided information also contains URLs to the code repositories, owners
of the code, the open-source status of the software packages, the size of user
communities and related information. Similar descriptions have been provided
for the application codes involved in the dealii-X CoE, including the
\href{https://lifex.gitlab.io/}{lifex},
\href{https://github.com/4C-multiphysics/4C}{4C},
\href{https://github.com/exadg/exadg}{ExaDG},
\href{https://github.com/fdrmrc/Polydeal}{polyDEAL}, LiverX and
\href{https://github.com/BRAINIACS-Group/ExaBrain}{ExaBrain} projects.

Furthermore, we have delivered details on the technical aspects of the codes
(size of project, dependencies, workflows, parallelization, supported
hardware, scalability, development plans), the porting status to various JU
systems in the European union, highlighting the focus of dealii-X on the
systems Lumi (porting to general GPU architecture), Leonardo (high priority),
and JUPITER (high priority). We also engage in exchange of ideas concerning
these systems. Furthermore, we have been open in checking for synergies in
various developments, and will actively interact with partners at various
scientific conferences. As a concrete step, the activities in WP2.6, with the
details listed in deliverable D2.6, the dealii-X project aims to work towards
a metascheduler for the simpler use of software across supercomputers. We will
seek active collaboration with other CoEs concerning the use of distributed
infrastructure and include available scheduling protocols developed within the
EU in the past as much as possible.

To foster this exchange, the dealii-X project will be represented in the
NCC-CoE-all-hands meeting in Tallinn, Estonia, in September 23--25, 2025.

The dealii-X consortium has been joining the \textbf{PMT-CoE leader meetings}
of CASTIEL 2, organized by the partners at HLRS (Stuttgart, Germany) since
November 2024, as collaborated on a joint Kokkos workshop on February 27, and
various other meetings.

Finally, we have provided \textbf{champions} and \textbf{deputy champions} for
the main areas of CASTIEL 2, namely the competences (WP2), the training (WP3),
the industry contacts (WP4) and communication (WP5).

\section{Collaboration on CI/CD}

An additional area of tight interaction is on continuous integration and
continuous deployment, where we have started interaction with the CI/CD team
of CASTIEL 2. We have been introduced into the existing infrastructure
established at EuroHPC. We are eager to use the upcoming months to interact
with the European Environment for Scientific Software Installations
(EESSI)\footnote{\url{https://www.eessi.io/}}, where we want to get our main
codes, starting from the core code \texttt{deal.II}, to be deployed at various
HPC systems. This collaboration with enable us to get continuous feedback on
the state of our software, which will add a crucial facet to the testing
infrastructure of deal.II. In addition, this is also of great benefit to our
users, as EESSI is subscribed to build a common stack of scientific software
installations for HPC systems as well as a variety of other systems.

The goal at this point is to interact with EESSI, to join the monthly meetings
as well as bi-weekly status updates, in order to get our main codes deployed
until October 2025. Initial, the coordination with CI/CD is done by dealii-X's
PI Martin Kronbichler. From April 2025, Ivan Pribec from the BADW-LRZ partner
will join the activities and coordinate with other CoEs. Overall, the
consortium is committed to increase the redundancies of duties on this topic
work to several partners, with primary focus on the partners from WP1 and WP2,
to ensure a high responsiveness and good progress.

\section{dealii-X Seminar Series}

To enhance collaboration and knowledge exchange among project partners, we have
initiated the development of an online seminar series showcasing cutting-edge
research by dealii-X collaborators and guest speakers. Launching in June 2025,
the seminars will follow a standard format of a 45-minute lecture followed by a
15-minute Q\&A session. Recordings of the sessions will be made available on the
project webpage and shared via social media channels. This initiative will be
carried out in close collaboration with the VPH Institute, which will assist in
editing and publishing the multimedia content (WP4).

Through the seminar initiative, we will also establish contacts to other
EuroHPC CoEs. It is planned that several talks will be given by researchers in
adjacent projects, giving the dealii-X members insights cutting-edge progress
of related topics. This setup will foster the collaboration between the
projects. As a platform to learn from other researchers and other projects,
the activities in the seminar series will boost the qualifications of our team
members and create opportunities for research results beyond the present state
of the art.

\section{Training and Workshops}

One of the consortium’s key goals is to enhance the training of clinical
researchers and practitioners through specialized workshops and interactive,
hands-on sessions. LRZ has a well-established mission to provide education on
programming and HPC-related topics. In response to growing demand, its course
offerings have expanded significantly in recent years to include GPU
programming, deep learning, and AI. These training sessions are promoted
through LRZ’s existing HPC network as well as EuroCC dissemination channels.

LRZ has actively contributed to several European projects, including
CompBioMed and LEXIS, where it played a key role in organizing meetings,
application-focused workshops, and hackathons (e.g., the preCICE workshop and
the CompBioMed-SEAVEA hackathon).

As part of the dealii-X consortium, the first workshop is scheduled for
December 2025, in Trieste, Italy, hosted by the partner SISSA. The event will
feature an open users and developers workshop, followed by project-specific
sessions addressing critical topics such as co-design, exascale computing, and
machine learning/deep learning. These sessions aim to strengthen partner
collaboration and foster multidisciplinary research.

\label{MyLastPage}


\end{document}
